% !TEX encoding = UTF-8
% !TEX program = pdflatex
% !TEX spellcheck = en_US

\documentclass[english,a4paper]{europasscv}
\usepackage[english]{babel}

\ecvname{Fernando Luz}

\ecvaddress{Vila Nova de Gaia - Portugal}
\ecvemail{prof.fernando.luz@gmail.com}
% \ecvhomepage{tbd}
% \ecvgitpage{www.git.example.org}
\ecvgithubpage{https://github.com/fluz}
% \ecvgitlabpage{www.gitlab.com/smith}
\ecvlinkedinpage{https://www.linkedin.com/in/f-luz/}
% \ecvorcid[label, link]{0000-0000-0000-0000}
% \ecvim{AOL Messenger}{katie.smith}
% \ecvim{Google Talk}{ksmith}

% \ecvgender{Female}
% \ecvdateofbirth{1 March 1975}
% \ecvnationality{Irish}

% \ecvpicture[width=3.8cm]{picture.jpg}

% \date{}

\begin{document}
  \begin{europasscv}

% BEGIN COVER LETTER =============

%     \thispagestyle{empty}
%     \eclpersonalinfo
%
%     \ecladdressee{Contact person}{Name of organization}{Address of organization}{City}
%
%     \eclcitydatesubject{City}{\today}{Subject of this cover letter}
%
%     \eclopeningsalutation{Dear Sir}
%     \eclmaincontent
%     {Opening salutation.\bigskip}
%
%     {This is the main content.}
%
%     {Closing salutation.}
%
%     \eclclosingsalutation{Yours sincerely}
%
%     \eclsignature
% %     \eclsignature[signature.jpg]
% %     \eclsignature[signature.jpg][Dr. Katie Smith (custom signature)]
%
%     \pagebreak
%     \setcounter{page}{1}


% BEGIN CV =============

  \ecvpersonalinfo

  \ecvbigitem{Position}{Engineering Manager}

  \ecvsection{Work experience}

  
  \ecvtitle{2022 - Present}{Engineering Manager}
  \ecvitem{}{Talkdesk}
  \ecvitem{}{Management of the \textbf{HLS --- Healthcare \& Life Science stream} and
\textbf{Support stream} at Industries BU. (manager of \textbf{10+}
Individual Contributors from entry levels to leads)

The HLS stream is responsible for \textbf{Healthcare Experience Cloud}
development, integrating healthcare call centers with unified patient
information.

The Support stream teams are composed of two squads. The \textbf{Cross
Industries} team implements tools and integrations to be used in other
projects in our BU. The \textbf{QA team} provides a framework and
procedure to validate the quality of our products.
}
  
  \ecvitem{Achievements:}{
  \begin{ecvitemize}
  
     \item Release of Agent Flows project (integration between EDP, Talkdesk and %
Zingtree) in Summer'22 release in 7 months (team creation, backlog %
organization and implementation); %

  
     \item Restructure the \textbf{QA implementation process}, providing some %
guidance how support the teams and organizing the backlog; %

  
     \item Define and implement the 1:1 weekly meetings and growth plans for each %
Indivudual Contributors and promote the link between each IC to the %
Talkdesk V2M2 (Vision, Value, Method and Metrics). %

  
  \end{ecvitemize}
  }
  
  
  \ecvtitle{2021 - 2022}{Senior Software Engineer}
  \ecvitem{}{Talkdesk}
  \ecvitem{}{Integrate a new team from the ground to create a new solution for
banking financial services.

The tech stack in this project uses \textbf{Kotlin (BE)},
\textbf{SpringBoot}, \textbf{React (FE)}, \textbf{Redis},
\textbf{PostGres SQL}, \textbf{MongoDB}, \textbf{RabbitMQ}.
}
  
  \ecvitem{Achievements:}{
  \begin{ecvitemize}
  
     \item Release the first verion of Visual IVR for Financial Services in the %
Summer release; %

  
     \item Created a in-house solution for Visua IVR Frontend; %

  
     \item Present some technical sessions to present new technologies (Functional %
Programming, K6). %

  
  \end{ecvitemize}
  }
  
  
  \ecvtitle{2018 - 2021}{Senior Software Engineer}
  \ecvitem{}{Capgemini Portugal [ASML Project]}
  \ecvitem{}{Responsible to start the \textbf{ASML} project, where I contributed in
\textbf{Variant Pattern} implementation at the robot component to be
used in next lithography machine generation (EXE-5000), and currently
I'm leading the Portugal team to expand unit tests quality using a
framework based on \textbf{gtest}. I was one of the founders in the
Meetup internal group in Altran PT.
}
  
  \ecvitem{Achievements:}{
  \begin{ecvitemize}
  
     \item Delivered first version for RYUN (Universal Pick and Place Robot) %
component, with \textbf{all features planned} for this version; %

  
     \item Added unit tests for RYUN component with \textbf{100\% of code %
coverage}; %

  
     \item Contributed with the RYAU component \textbf{migration} to legacy %
version; %

  
     \item Generated an investigation for RYAU autotesters, where I found %
\textbf{65\% fake tests} (tests without implementation); %

  
     \item Achieved with success the first phase for UTTK to ATTEST migration %
\textbf{(around 400 tests in 10 weeks)}. %

  
  \end{ecvitemize}
  }
  
  
  \ecvtitle{2016 - 2018}{IT Manager}
  \ecvitem{}{Technomar Engineering}
  \ecvitem{}{\textbf{Managed} the team responsible to implement new features in the
\textbf{TMS (Technomar Maritime Simulator)}. \textbf{TMS} is a vessel
maneuver simulator used in training activities. \textbf{Launched} the
certification planning for \textbf{TMS} simulator by
\href{https://www.dnvgl.com/}{DNV GL agency}, and the
\textbf{collaborated} in the the core of hydrodynamic numerical model
applied in a ``full bridge'' simulator. \textbf{Cultivated} good
practices in software development process, such as \textbf{Scrum/Kanban}
board, \textbf{TDD}, \textbf{Git} adoption and \textbf{mentored} meeting
sessions to promote the homogenization of the knowledge of the team.
}
  
  \ecvitem{Achievements:}{
  \begin{ecvitemize}
  
     \item Enhanced in communication channel using the phonon framework, with gains %
in \textbf{15\% speedup} and improve the code maintainability; %

  
     \item Redesigned the database in MongoDB, with a \textbf{definition of a %
schema} to be use in DB, and implemented a unified access library to %
access the data; %

  
     \item Orchestrated the \textbf{full delivery} of new simulation station at %
Technomar office, providing a simulator with 360 degrees immersion. %

  
     \item Instituted \textbf{Gitlab} as tool to obtain code metrics, and manage %
the bugs, new features, backlog and milestone control for the team and %
the founders. %

  
  \end{ecvitemize}
  }
  
  
  \ecvtitle{2008 - 2016}{Researcher / HPC Software Engineer}
  \ecvitem{}{Numerical Offshore Tank}
  \ecvitem{}{My role in the TPN laboratory was to develop a set of applications with
\textbf{High-Performance Computing}. The languages I use was
\textbf{C++/C}, \textbf{Python}, \textbf{MPI/sockets} and \textbf{bash}
in Linux and Windows environment. Other duties includes testing and
validation of numerical simulations, optimization, implementation of
parallel improvements and integration with other projects. I also worked
in others several projects, including the first version of
\textbf{vessel maneuver simulator} called \textbf{SMH}. This project was
select as finalist of \textbf{ANP (National Petroleum Agency)}
\href{http://www.anp.gov.br/pesquisa-desenvolvimento-inovacao/302-premio-anp-de-inovacao-tecnologica/edicoes-anteriores/888-premio-anp-de-inovacao-tecnologica-2016}{Prize
Award for Technological Innovation in 2016}.
}
  
  \ecvitem{Achievements:}{
  \begin{ecvitemize}
  
     \item Optimized parallel execution of the numerical solver in the cluster %
environment saving around \textbf{35\% in resources} using my %
Ph.D.~research; %

  
     \item Minimized around \textbf{10\% the time execution} in the standalone %
version of the numerical solver using OpenMP; %

  
     \item Reduced in order to \textbf{60\% the storage space} with the output %
results of the application using a binary format; %

  
     \item Championed and implemented a \textbf{cross compilation} in Windows and %
Linux environment with CMake; %

  
     \item Introduced a \textbf{MPI communication} instead pure socket, providing %
more robustness in the communication channel; %

  
     \item Championed the use of \emph{tests in development} flow to increase the %
development quality; %

  
     \item Built an initial \textbf{CI} with (CDash and CTest) to check the %
repository integrity. %

  
  \end{ecvitemize}
  }
  
  
  \ecvtitle{2011 - 2016}{Graduate Full Professor}
  \ecvitem{}{Paulista University}
  \ecvitem{}{Teaching-related responsibilities such as giving lectures, tutoring,
manage homeworks, laboratory activities, exams preparation and grading.
}
  
  

  \pagebreak

  \ecvsection{Education and training}

  
  \ecvtitle{2010 - 2015}{Ph.D. in Computing Engineering}
  \ecvitem{Title}{Methodology for execution of parallel applications based on BSP model with heterogeneous tasks}
  \ecvitem{}{Polytechnic School, University of São Paulo (USP) }
  
  \ecvtitle{2006 - 2010}{M.Sc. in Applied Physics}
  \ecvitem{Title}{Implementation of the MILC package in the study of full QCD}
  \ecvitem{}{Physics Institute of São Carlos, University of São Paulo (USP) }
  
  \ecvtitle{2001 - 206}{B.Sc. in Physics}
  
  \ecvitem{}{Physics Institute of São Carlos, University of São Paulo (USP) }
  


  \ecvsection{Personal skills}
  % \ecvmothertongue{English}
  % \ecvlanguageheader
  % \ecvlanguage{French}{C1}{C2}{B2}{C1}{C2}
  % \ecvlanguagecertificate{Diplôme d'études en langue française (DELF) B1}
  % \ecvlastlanguage{German}{A2}{A2}{A2}{A2}{A2}
  % \ecvlanguagefooter

  % \ecvblueitem{Communication skills}{
  % \begin{ecvitemize}
  %   \item team work: I have worked in various types of teams from research teams to national league hockey. For 2 years I coached my university hockey team
  %   \item mediating skills: I work on the borders between young people, youth trainers, youth policy and researchers, for example running a 3 day workshop at CoE Symposium ``Youth Actor of Social Change'', and my continued work on youth training programmes
  %   \item intercultural skills: I am experienced at working in a European dimension such as being a rapporteur at the CoE Budapest ``youth against violence seminar'' and working with refugees.
  % \end{ecvitemize}
  % }

  % \ecvblueitem{Organisational / managerial skills}{
  % \begin{ecvitemize}
  %   \item whilst working for a Brussels based refugee NGO ``Convivial'' I organized a ``Civil Dialogue'' between refugees and civil servants at the European Commission 20th June 2002
  %   \item during my PhD I organised a seminar series on research methods
  % \end{ecvitemize}
  % }

  % \ecvdigitalcompetence{\ecvBasic}{\ecvIndependent}{\ecvProficient}{\ecvIndependent}{\ecvBasic}

  \ecvblueitem{Technical skills}{
  \begin{ecvitemize}
    
    
    \item Agility
    
    
    
    \item Kotlin
    
    
    
    \item SpringBoot
    
    
    
    \item C++
    
    
    
    \item Bash
    
    
    
    \item Python
    
    
    
    \item Object-oriented design
    
    
    
    \item Jira
    
    
    
    \item C
    
    
    
    
    
    
    
    
    
    
    
    
    
    
    
    
    
    
  \end{ecvitemize}
  }

  \ecvblueitem{Soft skills}{
  \begin{ecvitemize}
    
    \item Team player
    
    \item Problem Solver
    
    \item Ownership
    
    \item Time management
    
    \item Build solutions
    
    \item Lead and deliver complex software systems
    
  \end{ecvitemize}
  }

  \ecvsection{Additional information}
  \ecvblueitem{VOLUNTEERING}{}
  
    \ecvtitle{Jun 2022 - Present}{President}
    \ecvitem{}{Associação de Pais e Encarregados de Educação da Escola Básica Manuel António Pina}
  
    \ecvtitle{Oct 2021 - Jun 2022}{Executive Secretary}
    \ecvitem{}{Associação de Pais e Encarregados de Educação da Escola Básica Manuel António Pina}
  

  \end{europasscv}

\end{document}