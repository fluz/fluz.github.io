%% Based on file `template.tex'.
%% Copyright 2006-2013 Xavier Danaux (xdanaux@gmail.com).
%
% This work may be distributed and/or modified under the
% conditions of the LaTeX Project Public License version 1.3c,
% available at http://www.latex-project.org/lppl/.


\documentclass[11pt,a4paper,sans]{moderncv}
% possible options include font size ('10pt', '11pt' and '12pt'), paper size ('a4paper', 'letterpaper', 'a5paper', 'legalpaper', 'executivepaper' and 'landscape') and font family ('sans' and 'roman')

\moderncvstyle{classic} % style options are 'casual' (default), 'classic', 'oldstyle' and 'banking'
\moderncvcolor{green} % color options 'blue' (default), 'orange', 'green', 'red', 'purple', 'grey' and 'black'

%\providecommand{\tightlist}{\setlength{\itemsep}{0pt}\setlength{\parskip}{0pt}}
\providecommand{\tightlist}{}
%\renewcommand{\familydefault}{\sfdefault} % to set the default font; use '\sfdefault' for the default sans serif font, '\rmdefault' for the default roman one, or any tex font name

\usepackage[utf8]{inputenc}

% adjust the page margins
\usepackage[scale=0.75]{geometry}
%\setlength{\hintscolumnwidth}{3cm}                % if you want to change the width of the column with the dates
%\setlength{\makecvtitlenamewidth}{10cm}           % for the 'classic' style, if you want to force the width allocated to your name and avoid line breaks. be careful though, the length is normally calculated to avoid any overlap with your personal info; use this at your own typographical risks...

% personal data
\name{Fernando Luz}{}
%\title{Resumé title}                               % optional, remove / comment the line if not wanted
%\address{street}{zip, city}{country}
\email{prof.fernando.luz@gmail.com}

\phone[mobile]{+351 91 955 2453}

\homepage{tbd}
\social[linkedin]{fluz}
\social[github]{fluz}
\social[twitter][www.twitter.com/fluz]{fluz}

%\extrainfo{additional information}
%\photo[64pt][0.4pt]{lpenz.png}
%\quote{Some quote}

\begin{document}

\makecvtitle

\section{About me}

\textbf{15+ years} of developing and working in projects led me to have
hands-on experience solving \textbf{complex problems} in software
development using several languages and frameworks across a wide range
of companies, including startups, oil and gas industries, educational
institutions, and research labs. Driven \textbf{self-starter} and
\textbf{fast learner}, always supporting my development teams with
\textbf{top-notch coding skills}. I enjoy the challenge to maintain
\textbf{team cooperation} to achieve objectives/deliveries in the
process development. I strongly believe that \textbf{sharing knowledge}
is a key point in a dev team, and I do my best to write clear and good
code and implement best practices.


\clearpage

\section{Experience}


\cventry{2022 - Present}{}{}{}{}{Management of the \textbf{HLS --- Healthcare \& Life Science stream} and %
\textbf{Support stream} at Industries BU. (manager of \textbf{10+} %
Individual Contributors from entry levels to leads) %
 %
The HLS stream is responsible for \textbf{Healthcare Experience Cloud} %
development, integrating healthcare call centers with unified patient %
information. %
 %
The Support stream teams are composed of two squads. The \textbf{Cross %
Industries} team implements tools and integrations to be used in other %
projects in our BU. The \textbf{QA team} provides a framework and %
procedure to validate the quality of our products. %
}

\cventry{2021 - 2022}{}{}{}{}{Integrate a new team from the ground to create a new solution for %
banking financial services. %
 %
The tech stack in this project uses \textbf{Kotlin (BE)}, %
\textbf{SpringBoot}, \textbf{React (FE)}, \textbf{Redis}, %
\textbf{PostGres SQL}, \textbf{MongoDB}, \textbf{RabbitMQ}. %
}

\cventry{2018 - 2021}{}{}{}{}{Responsible to start the \textbf{ASML} project, where I contributed in %
\textbf{Variant Pattern} implementation at the robot component to be %
used in next lithography machine generation (EXE-5000), and currently %
I'm leading the Portugal team to expand unit tests quality using a %
framework based on \textbf{gtest}. I was one of the founders in the %
Meetup internal group in Altran PT. %
}

\cventry{2016 - 2018}{}{}{}{}{\textbf{Managed} the team responsible to implement new features in the %
\textbf{TMS (Technomar Maritime Simulator)}. \textbf{TMS} is a vessel %
maneuver simulator used in training activities. \textbf{Launched} the %
certification planning for \textbf{TMS} simulator by %
\href{https://www.dnvgl.com/}{DNV GL agency}, and the %
\textbf{collaborated} in the the core of hydrodynamic numerical model %
applied in a ``full bridge'' simulator. \textbf{Cultivated} good %
practices in software development process, such as \textbf{Scrum/Kanban} %
board, \textbf{TDD}, \textbf{Git} adoption and \textbf{mentored} meeting %
sessions to promote the homogenization of the knowledge of the team. %
}

\cventry{2008 - 2016}{}{}{}{}{My role in the TPN laboratory was to develop a set of applications with %
\textbf{High-Performance Computing}. The languages I use was %
\textbf{C++/C}, \textbf{Python}, \textbf{MPI/sockets} and \textbf{bash} %
in Linux and Windows environment. Other duties includes testing and %
validation of numerical simulations, optimization, implementation of %
parallel improvements and integration with other projects. I also worked %
in others several projects, including the first version of %
\textbf{vessel maneuver simulator} called \textbf{SMH}. This project was %
select as finalist of \textbf{ANP (National Petroleum Agency)} %
\href{http://www.anp.gov.br/pesquisa-desenvolvimento-inovacao/302-premio-anp-de-inovacao-tecnologica/edicoes-anteriores/888-premio-anp-de-inovacao-tecnologica-2016}{Prize %
Award for Technological Innovation in 2016}. %
}

\cventry{2011 - 2016}{}{}{}{}{Teaching-related responsibilities such as giving lectures, tutoring, %
manage homeworks, laboratory activities, exams preparation and grading. %
}


\clearpage

\section{Education}


\cventry{2015}{}{}{}{}{}

\cventry{2010}{}{}{}{}{}

\cventry{2006}{}{}{}{}{}



\section{Other courses}




\section{Languages}



\clearpage

\section{Random achievements and anecdotes}



\end{document}

% vim: ft=tex.jinja